\subsection{Nearest-Neighbor Calculation for Iris Dataset}
\subsubsection{Preprocessing}
The Iris dataset contains four attributes, each being of the ratio type. For preprocessing each attribute was normalized, i.e centered on zero and the standard deviation was reduced to 1.
\subsubsection{Proximity Functions}
Each record of the dataset is denoted as $A = {a_1, a_2, a_3, a_4}$.
Two different proximity functions $P(A,B)$ were used -
\begin{itemize}
	\item Euclidean Distance
		\begin{equation}
			P(A,B) = \left(\sum_{i=1}^{4} a_i^2 + b_i^2\right)^{\frac{1}{2}}
		\end{equation}
	\item Cosine Distance
		\begin{equation}
			P(A,B) = \frac{(A\cdot B)}{\lVert A\rVert \lVert B\rVert}
		\end{equation}			
\end{itemize}
\subsection{Nearest-Neighbor Calculation for Income Dataset}

\subsubsection{Preprocessing}
\begin{enumerate}
\item The \emph{ID} attribute was dropped as it is unique for each record. The role of \emph{fnlwgt} was not discernible so it too was dropped.
\item The Income dataset has three kinds of attributes - Nominal, Ordinal and Ratio. The dataset was split into three as the three kinds of attributes had to be treated differently.
\item Each of the ratio attributes were normalized by centering the mean to zero and standard deviation to 1.
\item The ordinal attributes were also normalized to mean 0 and standard deviation 1.
\item The nominal attributes were left as they were.
\end{enumerate}
\subsubsection{Proximity Functions}
For the nominal attributes a simple distance measure was used - 
\begin{equation*}
d^{(12)}_i = \begin{cases} 
      0 & \textrm{ if $A_i^{(1)} = A_i^{(2)}$} \\
      1 & \textrm{ if $A_i^{(1)} \neq A_i^{(2)}$} \\
   \end{cases}
\end{equation*}
where $A^{(k)}_i$ represents the $i^{th}$ nominal attribute of the $k^{th}$ record. The total distance for between $A^{(1)}$ and $A^{(2)}$ is given by 
\begin{equation*}
	d^{(12)} = \sum^n_{i=1} d^{(12)}_i\\
\end{equation*}
where the the total number of nominal attributes is $n$. \\
For the ratio attribute type, the distances were calculated using euclidean and cosine dissimilarity as defined in equation $(1)$ and $(2)$. And finally for the ordinal data the distance was computed by calculating the absolute difference -
\begin{equation*}
	d^{(12)}_i = |A_i^{(1)} - A_i^{(2)}|
\end{equation*}
Finally these three distances were combined by taking the weighted average based on the number of attributes in each -
\begin{equation*}
	d^{(12)}_{\textrm{total}} = \frac{n_{\textrm{nominal}}d^{(12)}_{\textrm{nominal}} + n_{\textrm{ordinal}}d^{(12)}_{\textrm{ordinal}} + n_{\textrm{ratio}}d^{(12)}_{\textrm{ratio}}}{n_{\textrm{nominal}} + n_{\textrm{ordinal}} + n_{\textrm{ratio}}}
\end{equation*}

\subsubsection{Handling Missing Values}
The missing values occur only for nominal attributes. The distance from such a record was taken as 1 for any record, even if the record in question too has a missing value in that place. The method of computation does not make any assumtion about the possible value of the 'Missing Value'. So two records that have a missing value for the attribute have a distance of 1 due to that attribute. And since it is missing it is definitely different from know values, the distance will be 1.\\


