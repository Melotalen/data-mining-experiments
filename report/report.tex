%% Neural Networks For Classification
%% 2014/07/10
%% by Soren Goyal
%%-----------------------------------------------------------------------------------------------------
\documentclass[journal,transmag]{IEEEtran}

\usepackage{cite}
\usepackage[cmex10]{amsmath}
\usepackage{amssymb}
\usepackage{algorithmic}
\usepackage{array}
\usepackage{graphicx}
	%\graphicspath{images/}
\usepackage{url}

\begin{document}

\title{CSE 5243\\Homework 2}
\author{\IEEEauthorblockN{Soren Goyal, Ohio State University}}

%\IEEEtitleabstractindextext{
%\begin{abstract}
%Recognition of objects using Deep Neural Networks is an active area of research and many breakthroughs have been made in the last few years. The paper attempts to indicate how far this field has progressed. The paper briefly describes the history of research in Neural Networks and describe several of the recent advances in this field. The performances of recently developed Neural Network Algorithm over benchmark datasets have been tabulated. Finally, some the applications of this field have been provided.
%\end{abstract}
% Note that keywords are not normally used for peer review papers.
%\begin{IEEEkeywords}
%Convolutional, Neural Networks, Datasets, ILSVRC, Pooling, Activation Functions, Regularization, Object Recognition, Datasets
%\end{IEEEkeywords}}

% make the title area
\maketitle
\section{Section 1: Exploratory Analysis of Income Dataset}
	The main goal of this assignment was to understand the behavior of the kNN classfier using metrics from ROC curves, confusion matrix, etc.
The kNN classifier was applied to the two datasets - Iris and Income. More detailed analysis was done for Income Dataset.
\subsection{Datasets}
\begin{itemize}
\item \textbf{Iris Dataset} - The Iris Dataset has four attributes. Each attribute is numeric. It has 150 records, distributed equally among 3 classes - \emph{Setosa, Versicolor and Virginia}.
\item \textbf{Income Dataset} - The Income Dataset has 15 attributes, which are a mix of nominal, numeric and ordinal data. There are a total of 520 records in the training set and 288 in the test set. There are only two classes. The ratio between the classes is approximately $1:3$ with the negative class being larger in number.
\end{itemize}
\subsection{KNN Algorithm}
The prediction of KNN algorithm was broken down into main functions. The first one was \texttt{computeDistances(train\_data, test\_data)} and the second one was \texttt{predictLabel(k, distances)}.
\begin{algorithm}
	\begin{algorithmic}
	\Procedure{computeDistances}{$train\_data, test\_data$}
		\For{i in test\_data }
			\State distance[] = zeros()
			\For{j in train\_data}
				\State distance[j] $\gets$ computeDistance(r,s)
			\EndFor
			\State distances[i,j] $\gets$ sort(distance, order = increasing)
		  \State labels[i,j] $\gets$ getLabels(distance) \Comment{\emph{Gives the training data labels of the classes sorted according to distance from test record}}
		  \State \textbf{return} distances, labels
		\EndFor
	\EndProcedure
	\end{algorithmic}
\end{algorithm}
\begin{algorithm}
	\begin{algorithmic}
	\Procedure{predictLabel}{$k, distances, labels$}
		\State weight[] = {0,0}
		\For{i $in$ 1:nrow(distances)}
			\For{i $in$ 1:k}
				weight[labels[i,j]] = weight[labels[i,j]] + 1/distances[i,j]
			\EndFor
			\State	prob[i] = weight[2]/sum(weights)
		\EndFor
	\EndProcedure
	\end{algorithmic}
\end{algorithm}
To classfiy a new records the \texttt{predictLabel(k, distances)}

The \texttt{computeDistances()} function computes the distance of each of the test record from each of the training data record. The distances are sorted and stored for use by \texttt{predictLabel(k, distances)}. \texttt{predictLabel(k, distances)} takes the number of nearest neighbors (k) as a parameter and predicts the label. The detailed defintions of the functions are given in fig. ?? and fig. ??. \\

The reason for separating out the distance computation was the predictions were to be made for different values of $k$. Distance computation takes $O(mn\log n)$ time, where m is the number of records in test dataset, n is the number of records in training dataset. While prediction takes only $O(mk)$ time, which is considereably less than $O(mn\log n)$.\\
To classify the test records, the KNN algorithm was used. A number of variations were tried in each of the algorithms. This section explains each of the variations and discusses the pros and cons of each.
\subsubsection{Variation in Preprocessing}
 \begin{itemize}
 	\item \textbf{Simple Scaling} - The nominal attributes are not touched. Numeric and ordinal attributes are scaled and centered. The resultant standard deviation was 1 and mean was 0.
 	\item \text{Simple Scaling with Binning} - Apart from scaling and centering the numeric and ordinal attributes the nominal attriutes were also modified. Many of the nominal attributes were very skewed. For e.g - In \emph{Native Country} in Income dataset, out of 26 categories, 12 categories had only one record each. In such cases the smalled categories were grouped together. Also \emph{Education\_cat} was converted to nominal type by grouping a few educational levels what had very few records.
 \end{itemize}
 \subsubsection{Variations in Proximity Metric}
 \begin{itemize}
 	\item \textbf{Euclidean Distance-Simple Dissimilarity} - For numeric attributes euclidean distances were computed, while for nominal attributes simple dissimilarity was computed. These were combined by taking a weighted average.
 	\item \textbf{Cosine Similarity-Simple Dissimilarity} - Nominal atributes were treated the same way as the previous type. But for numeric attributes the cosine similarity was computed. Finally, they were combined in the same way as the previous method i.e by taking a weighted average.
 \end{itemize}
\subsubsection{Algorithm for Voting}
 	 	Let $N$ represent the set of $k$ nearest neighbors. Let $N^(1)\subseteq N$ be set of neighbors belonging to class 1 and $N^(2)$ be the set of neighbors belonging to the other class.
\textbf{Inverse Distance Weighted Voting}: Once the $k$ nearest neighbours have been determined, the probablity of a point belonging to a class is as follows - 
\begin{equation*}
	P(class = i) = \frac{\sum_{r \in N^(i)} \frac{1}{distance(r)}}{\sum_{r \in N} \frac{1}{distance(r)}}
\end{equation*}
Finally, the class was assigned based on a threhold that can be set to get the target true positive rate or false psoitive rate.

\section{Section 2: Description of Programs}
	\subsection{Section II: Analysis of Performance of Classifiers}
\subsubsection{Classifier Performance for Iris Dataset}
	\textbf{Classifier 1}\\
	\begin{itemize}
		\item Preprocessing - Simple Scaling
		\item Proximity Metric - Euclidean Distance
		\item Voting - Inverse Distance Weighted Voting
	\end{itemize}
	The confusion matrix for three different values of k is given in Fig: ?
	\begin{figure}[h]
			\label{fig:iris_k=1}
			\caption{Confusion Matrix k = 1 (Datset:Iris Classifier:1)}
			\centering
			\includegraphics[width=0.3\textwidth]{images/iris_k1.png}
	\end{figure}
	\begin{figure}[h]
		\label{fig:iris_k=5}
		\caption{Confusion Matrix k = 5 (Datset:Iris Classifier:1)}
		\centering
		\includegraphics[width=0.3\textwidth]{images/iris_k=5.png}
	\end{figure}
	\begin{figure}[h]
		\label{fig:iris_k=10}
		\caption{Confusion Matrix k = 10 (Datset:Iris Classifier:1)}
		\centering
		\includegraphics[width=0.3\textwidth]{images/iris_k=10.png}
	\end{figure}	
	??Analysis??\\
	
	\textbf{Classifier 2}\\
	\begin{itemize}
		\item Preprocessing - Simple Scaling (same as before)
		\item Proximity Metric - Cosine Similarity
		\item Voting - Inverse Distance Weighted Voting (same as before)
	\end{itemize}
	The confusion matrix is given in Fig: ??(use excel)
	\begin{figure}[h]
			\label{fig:iris_cosine_k=1}
			\caption{Confusion Matrix k = 1 (Datset:Iris Classifier:2)}
			\centering
			\includegraphics[width=0.3\textwidth]{images/iris_cosine_k=1.png}
	\end{figure}
	\begin{figure}[h]
		\label{fig:iris_consine_k=5}
		\caption{Confusion Matrix k = 5 (Datset:Iris Classifier:2)}
		\centering
		\includegraphics[width=0.3\textwidth]{images/iris_cosine_k=5.png}
	\end{figure}
	\begin{figure}[h]
		\label{fig:iris_cosine_k=10}
		\caption{Confusion Matrix k = 10 (Datset:Iris Classifier:2)}
		\centering
		\includegraphics[width=0.3\textwidth]{images/iris_cosine_k=10.png}
	\end{figure}
	??Analysis??

\subsubsection{Classifier Performance for Income Dataset}
To analyze the performance of kNN on the Income dataset two parameters need to be determined. The number of nearest neighbours $k$ and threshold for making the prediction. To optimum $k$ for the classifier would be the $k$ for which the ``Area Under the Roc Curve" is maximized. To do this, the roc graphs were computed over the test set for $k$'s ranging from 1 to 100. The Area under the curve increases as k increase. This means that the classifer is getting better at all levels. However after a while it peaks and then plateus. The $k$ at the peak can be chosen as of the best $k$'s.

\textbf{Classifier 1}\\
	\begin{itemize}
		\item Preprocessing - Simple Scaling
		\item Proximity Metric - Euclidean Distance
		\item Voting - Inverse Distance Weighted Voting
	\end{itemize}
	The confusion matrix is given in Fig: ??(use excel)\\
	The ROC curve is given in Fig: ??\\
	For this classifier the Area under ROC reaches maximum at k = 38 \\
	\begin{figure}[h]
		\label{fig:classifier1_auc}
		\caption{Plot of Area Under the Roc curve for different values of k for Income Classifier 1. Maxima is achieved at k = 36}
		\centering
		\includegraphics[width=0.3\textwidth]{images/income_classifier1/auc.jpg}
	\end{figure}	
	\begin{figure}
		\label{fig:classifier1_roc}
		\caption{ROC curves for different values of k for Income Classifier 1}
		\centering
		\includegraphics[width=0.3\textwidth]{images/income_classifier1/roc.jpg}
	\end{figure}
	\begin{figure}
		\label{fig:classifier1_accuracy}
		\caption{Plot of Accuracy Vs Threshold Values for k = 38}
		\centering
		\includegraphics[width=0.3\textwidth]{images/income_classifier1/accuracy.jpg}
	\end{figure}	
	
\textbf{Classifier 2}\\
	\begin{itemize}
		\item Preprocessing - Simple Scaling
		\item Proximity Metric - Cosine Similarity
		\item Voting - Inverse Distance Weighted Voting
	\end{itemize}
	The confusion matrix is given in Fig: ??(use excel)\\
	The ROC curve is given in Fig: ??\\
	Max for k = 36??Analysis??\\
	\begin{figure}[h]
		\label{fig:classifier2_auc}
		\caption{Plot of Area Under the Roc curve for different values of k for Income Classifier 2. Maxima is achieved at k = 36}
		\centering
		\includegraphics[width=0.3\textwidth]{images/income_classifier2/auc.jpg}
	\end{figure}	
	\begin{figure}
		\label{fig:classifier2_roc}
		\caption{ROC curves for different values of k for Income Classifier 2}
		\centering
		\includegraphics[width=0.3\textwidth]{images/income_classifier2/roc.jpg}
	\end{figure}
	\begin{figure}
		\label{fig:classifier2_accuracy}
		\caption{Plot of Accuracy Vs Threshold Values for k = 36}
		\centering
		\includegraphics[width=0.3\textwidth]{images/income_classifier2/accuracy.jpg}
	\end{figure}
	
\textbf{Classifier 3} - Best Performance\\
	\begin{itemize}
		\item Preprocessing - Simple Scaling with Binning
		\item Proximity Metric - 
		\item Voting - Inverse Distance Weighted Voting with Positive Class Priority
	\end{itemize}
	The confusion matrix is given in Fig: ??(use excel)\\
	The ROC curve is given in Fig: ??\\
	??Analysis??\\
	\begin{figure}[h]
		\label{fig:classifier3_auc}
		\caption{Plot of Area Under the Roc curve for different values of k for Income Classifier 3. Accuracy keeps getting beyond hundred}
		\centering
		\includegraphics[width=0.3\textwidth]{images/income_classifier3/auc.jpg}
	\end{figure}	
	\begin{figure}
		\label{fig:classifier1_roc}
		\caption{ROC curves for different values of k for Income Classifier 3}
		\centering
		\includegraphics[width=0.3\textwidth]{images/income_classifier3/roc.jpg}
	\end{figure}
	\begin{figure}
		\label{fig:classifier3_accuracy}
		\caption{Plot of Accuracy Vs Threshold Values for k = 40}
		\centering
		\includegraphics[width=0.3\textwidth]{images/income_classifier3/accuracy.jpg}
	\end{figure}

\section{Section 3: Analysis of Nearest Neighbors}
	The characteristics of the nearest neighbors of each record can provide useful insight into the nature of clustering of data.

\subsection{Variation of distribution for the first four nearest neighbors}
The change in the distribution of the distance of nearest neighbor with the class and neighbor position, is shown in Fig:11 and Fig:12. The method to generate the plot is as follows - 
\begin{figure}[h]
		\label{fig:proximity-distribution-class1}
		\caption{Distribution of distances of the first 4 nearest neighbors for records in class '$<=$50K'}
		\centering
		\includegraphics[width=0.5\textwidth]{images/proximity-distribution-class1.jpg}
\end{figure}
\begin{figure}[h]
		\label{fig:proximity-distribution-class2}
		\caption{Distribution of distances of the first 4 nearest neighbors for records in class '$>$50K'}
		\centering
		\includegraphics[width=0.5\textwidth]{images/proximity-distribution-class2.jpg}
\end{figure}

\begin{enumerate}
\item The dataset was divided in to two by class.
\item For each element in each class, the first four nearest neighbors and their distances were computed.
\item The distributions were plotted for each class separately. The distance of the neighbor increases along the positive x-axis, while the y-axis represents the number density. In mathematical terms, the number of records having a neighbor between distances $x$ and $x+dx$ is given by $ydx$.
\end{enumerate}

For the lower income class (Fig:11) it can be seen that the distributions for the four nearest neighbors has two peaks. The first peak is at 0.2-0.3 and the second peak is at 0.8 - 0.9. These two peaks represent two clusters. In the first cluster the points belonging to lower income class are clustered close together with most of the points having their first neighbor at 0.2-0.3. The second cluster is larger but sparse. The first neighbor in this cluster is at 0.8-0.9 for most of the points. Also the the distributions move rightward as we move towards the fourth neighbor. However the change is not so much for the third and fourth nears neighbors. This indicates a saturation of some kind.\\

For the upper income class (Fig:12) the four distributions are different from the lower income class. There is only one peak which gradually widens as we move to distributions for higher number of neighbors. This indicates the presence of only one cluster in the upper income class. Also this data is more closely packed as compared to lower income dataset. This indicates that the people in upper income bracket have much less variation in their attributes as compared to people in lower income category.

\subsection{Variation of distribution of number of neighbors in a 1 unit ball}
The variation of distribution of the number records in 1 unit proximity of a record is plotted in Fig:\ref{fig:numberof-neighbors-class1} and Fig:\ref{fig:numberof-neighbors-class2}. The method to generate the plot is as follows - 

\begin{figure}[h]
		\label{fig:numberof-neighbors-class1}
		\caption{Distribution of distances of the first 4 nearest neighbors for records in class '$<=$50K'}
		\centering
		\includegraphics[width=0.5\textwidth]{images/numberof-neighbors-class1.jpg}
\end{figure}
\begin{figure}[h]
		\label{fig:numberof-neighbors-class2}
		\caption{Distribution of distances of the first 4 nearest neighbors for records in class '$>$50K'}
		\centering
		\includegraphics[width=0.5\textwidth]{images/numberof-neighbors-class2.jpg}
\end{figure}

\begin{enumerate}
\item The dataset was divided in to two according to the class.
\item For records of each class the number of neighbors in a unit ball were calculated. These calculation were made across the classes and with in the classes.
\item Each plot has the number of neighbors on the x-axis and the density on the y-axis.
\item Two distributions were obtained for each of the classes. For the upper income class the distribution in red shows the distribution of number of lower income people in a unit ball neighborhood. The green line in the same plot shows the distribution of other upper income people in the same sized neighborhood. The other plot similarity shows the two distributions for the lower income people.
\end{enumerate}

In the plot for lower income people (Fig:13) there is a massive peak in the green line. The peak is very close to zero. This means that for large number of lower income people there close to 0 or 1 upper income points in their neighborhood. The red graph has a small bump that is spread over a number of values from 0 to 4, indicating that for a large number of points in lower income class the space is sparsely populated with 1-4 neighbors. The tail of green class extends to 30, which indicates the presence of outliers in the lower income class that is deep inside the upper income class territory. The tail of the red graph indicates that the there are over 30-40 records in lower income category that are more or less the same.\\
The distributions are a little different for the records in upper income category. The red graph has a peak closer to zero. So many of the upper income records do have a lower income record in their neighborhood. Instead they have each other in their neighborhood. This is indicated by the green graph which has a peak over 1. The two peaks are not very pronounced, hence it can be concluded that the many of the upper income records are rather spread out and are at times quite close to the lower income records.\\

This analysis gives a good idea of what 'k' to choose for a k-NN classifier. Since most records have 2 or more of the same kind of record in their neighborhood, it can be expected that a k-NN with k=2 will give a good classification results.\\

%	\input{parts.tex}
%\section{Data Sets used for Evaluation}\label{sec:Datasets}
%	\input{dataset.tex}
%\section{Performances of Neural Networks}\label{sec:Performances}
%	\input{performance.tex}
%\section{Emerging Applications}
%	\input{emerging-applications.tex}
%\section{Conclusion}\label{Conclusion}
%	\input{conclusion.tex}

\end{document}


